\documentclass[10pt]{article}
\usepackage{NotesTeXV3,lipsum}


\title{Hermit and Unitary Matrix in Quantum Mechanics}
\author{Hyunseong Kim}

\begin{document}
\maketitle
\newpage
\pagestyle{fancynotes}

\section{State of the system}

In physics, a state function is a quantity function that represents state of the system. 
Not only in quantum mechanics, and also in statistical or thermodynamics, 
this function is commonly treated as a system itself and used in many equations.

In quantum system, the state function is a complex valued wave function, $| \psi \rangle$.
We assume that all information we want from the system can be derived from the wave function by the measurement. 

The $| \rangle$ is a Dirac's braket notation. $| \rangle$ means vector and $\langle |$ means dual-vector.

Following Copenhagen interpretation, $| \psi \rangle$ yields a probability for the outcomes of measurements upon the system.

The probabilitic treatment constrains $\psi$ in some manners.
First, it must be a normalized function. 
Second, the measurement outcome must be \textit{real} value.

\begin{itemize}
    \item $\langle \psi| \psi\rangle = \int_{V} \psi(\mathbf{x})^{\*}| \psi(\mathbf{x}) \, \mathbf{dx} = 1$
    \item Measurement of quantity, $H$ on the system $| \psi \rangle$ is a $\langle \psi|\hat{H}| \psi\rangle $.
\end{itemize}

\section{Hermit and Unitary matrix}

\subsection{Unitary matrix}

Let's think about there is a change, whatever it is, in the system.
The $| \psi \rangle$ represent all the information of the system, so that it will be changed to $| \psi' \rangle$. 

Any modification in the vector space can be represented with an \textit{operator}, $\hat{U}$.

\begin{equation}
    |\psi' \rangle = \hat{U} | \psi \rangle
\end{equation}

Now, the modified state function also satisfies normalization, such as $\langle \psi' | \psi \rangle = \langle \psi | \psi \rangle$.

\begin{eqnarray*}
    \langle \psi' | \psi \rangle = \langle \hat{U} \psi | |\hat{U} \psi \rangle \\ 
    \langle \hat{U} \psi | |\hat{U} \psi \rangle = \langle  \psi| \hat{U}^{\dagger}|\hat{U} \psi \rangle\\
    \langle \psi| \hat{U}^{\dagger}|\hat{U} \psi \rangle = \langle \psi| \hat{U}^{\dagger}\hat{U}| \psi \rangle
\end{eqnarray*}

we get,

\begin{equation}
    \label{eq:unitary}
    \hat{U}^\dagger \hat{U} = \hat{I}
\end{equation}

where, $\hat{I}$ is an identity operator. 

That means that any state change event in the quantum system must be a unitary operator in vector space, in isolated system.
With well defined basis, we can formulate the operator as matrix, 

\begin{eqnarray*}
    {|\Psi \rangle} &= {\sum c_i |\psi_i \rangle} \\
    {[\hat{U}]}_{\psi_i} &= \sum (\langle \psi_j | \hat{U} |\psi_i \rangle) |\psi_i \rangle \langle \psi_j|
\end{eqnarray*}

It is little bit weired that the function operation as a matrix, however, we are treating basis function that generating all functions.
About the those set of functions we can find well-ordered basis, of course it does not have to be finite.
Even in the infinite dimensional vector space, we can find a subspace consist of discreted index basis.
Think about the Fourier series of the $L$ periodic function. 
The basis functions are $\cos(\lambda_n x), n \in \mathbb{Z}_+ $.

That is why the unitary matrix is essential topic in quantum computation and simulation.

\subsubsection{Properties of unitary matrix}

\begin{itemize}
    \item It preserves the inner product of two vector, $\mathbf{x,y}, \langle \mathbf{x |y }\rangle = \langle U \mathbf{x}| U \mathbf{y}\rangle$
    \item It is a normal operator: $A A^\dagger = A^\dagger A$.
    \item $U^\dagger = U^{-1}$
    \item There exists a Hermit matrix $H$ such that $U = \exp(i H)$.
    \item Eigenvalues are \textit{unimodular} which is their norms are 1. Therefore, $|\det(U)| =1$.
\end{itemize}

In quantum systems symulation, finding a proper unitary operation on system is important work finding equivalence but less cost unitary matrices exist in many cases.

In the property of the unitary matrix, there is $U = \exp(i H)$. 
It is a very familiar term in quantum mechanics; time-evolution operator. 

\subsection{Hermit matrix}

Suppose the Hamiltonian of the system is given as $H$. The Schr$\ddot{\mbox{o}}$dinger  equation yields next.

\begin{equation}
    i \hbar \frac{d}{d t} | \psi \rangle = H | \psi \rangle
\end{equation}

Hamiltonian is a kind of operator of measurement for energy of the system. It means that the eigenvalues of matrix are energy of the eiegenstates.
Such that 

\begin{equation}
    \hat{H} | \psi \rangle = E | \psi \rangle
\end{equation}


\begin{eqnarray}
    \langle \psi | \hat{H} | \psi \rangle = \langle \psi | |\hat{H}\psi \rangle = \langle \hat{H} \psi | | \psi \rangle \\
    \langle \psi |\hat{H}^\dagger | \psi \rangle = E^{\*} \langle \psi | \psi \rangle = E \langle \psi | \psi \rangle \\
    \therefore E^{\ast} = E
\end{eqnarray}

$E^{\ast} = E, \forall E$, the only complex value satisfying this constraint is a real value.
It means that the all eigenvalues of the matrix are real value.
Such matrices are called self-adjoint matrix or \textit{Hermit matrix}.

The definition of self-adjoint matrix is 

\begin{equation}
    H^\dagger = H .
\end{equation}

It is equivalence to the all eigenvalues are real condition.

We refered a Hamiltonian as an example of measurement, 
however, any measurement quantity operators are represented with Hermit matrices.

\subsubsection{Properties of Hermite matrix}

\begin{itemize}
    \item All eigenvalues are real value.
    \item It is a self-adjoint matrix.
    \item All eigenvector having different eigenvalues are orthogonal to each other.
    \item Normal matrix.
    \item Closed under addition.
    \item If the two Hermite matrices are commute each other, then their product is Hermite matrix.
\end{itemize}

Additional topic: time evolution operator of Hamiltonian,

\end{document}