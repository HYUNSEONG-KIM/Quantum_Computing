\section{Circuit representation of the evolution operator}

This section is about a practical implementation of the evolution operator.
From the above sections, we overlooked some basic decomposition techniques of 
evolution operators of general quantum systems.

\subsection{Pauli-matrices and decomposition}

The general convention of Hamiltonian evolution operator is 
an inner product to Pauli-vector, $\mathbf{\hat{\sigma}}$.
\begin{equation}
    \mathcal{H} = \mathcal{H} \cdot \mathbf{\hat{\sigma}}
\end{equation}.
In 2-dimension system, it becomes $[\hat{\sigma_X}, \hat{\sigma_Y}, \hat{\sigma_Z}]^T$.
\begin{equation}
    \label{eq:pauli-matrix}
    \sigma_X = \begin{pmatrix}
        0 & 1\\
        1 & 0
    \end{pmatrix}, \,
    \sigma_Y = \begin{pmatrix}
        0 & -i\\
        i & 0
    \end{pmatrix}, \,
    \sigma_Z = \begin{pmatrix}
        1 & 0\\
        0 & -1
    \end{pmatrix}
\end{equation}
\index{Pauli!Polynomial}
Precisely, such convention is a Pauli-polynomial as shown in Eq (\ref{eq:PauliDecompositon}).
Now, how did the general evolution operators implemented on quantum circuit?
It becomes rotations on specific axes of the given Hilbert space.
With the Pauli-polynomial representation of the given Hamiltonian,
the next relationship is hold true, see proof in Appendix \ref{AppendixProof01}.
\begin{equation}
    \label{eq:hamiltonain-exponential}
    \exp(-i \theta \mathcal{H} \cdot \hat{\sigma}) 
    = \cos(\theta ||\mathcal{H}||) \hat{I} - i \sin(\theta ||\mathcal{H}||) \hat{\mathcal{H}} \cdot \hat{\sigma}
\end{equation}

For general $2^n$-dimension system, it is enough to show that the construction rule of every Pauli-$n$ strings.
We first look at single qubit cases and we are going to expand it to the general $n$-qubit system.

\subsection{Evolution operator on single qubit system}

The Pauli-string of 1-qubit system are $\{X, Y, Z\}$.

\begin{equation}
    \mathcal{H} = \lambda_{X} X + \lambda_{Y} Y + \lambda_{Z} Z
\end{equation}

Therefore, the evolution operator of total Hamiltonian consists of 
3 rotation gates.

\begin{enumerate}
    \item $\exp(-i t_{X} X) = \cos(t_{X}) \hat{I} - i \sin(t_{X}) \hat{X} = RX(2t_{X})$
    \item $\exp(-i t_{Y} Y) = \cos(t_{Y}) \hat{I} - i \sin(t_{Y}) \hat{Y} = RY(2t_{Y})$
    \item $\exp(-i t_{Z} Z) = \cos(t_{Z}) \hat{I} - i \sin(t_{Z}) \hat{Z} = RZ(2t_{Z})$
\end{enumerate}

Be aware that they are not commute each other. 
You must use Product or ST2 formula when you deal with them simultaneously.

We have 3 rotation gates by the axis, but we can generate the other two rotation with one rotation gate and 
transformation gates. 
See an eigen decomposition form of each matrix.

\begin{equation}
    A = Q D Q^{\dagger}
\end{equation}
where, $Q = [e_1, e_{-1}]$, $e_{\lambda}$ is an eigenvector corresponding to eigenvalue, $\lambda$.
Since, $\sigma_{i}, i \in [X, Y, Z]$ have same eigenvalues, $1, -1$, $D = Z$.
The eigenvector of $X, Y$ are

\begin{itemize}
    \item $X$: $\frac{1}{\sqrt{2}} \begin{pmatrix}
        1 \\
        1
    \end{pmatrix}, \frac{1}{\sqrt{2}} \begin{pmatrix}
        1 \\
        -1
    \end{pmatrix}$
    \item $Y$: $\frac{1}{\sqrt{2}} \begin{pmatrix}
        1 \\
        i
    \end{pmatrix}, \frac{1}{\sqrt{2}} \begin{pmatrix}
        1 \\
        -i
    \end{pmatrix}$
\end{itemize}
then,
\begin{align}
    Q_{X} = \frac{1}{\sqrt{2}} \begin{pmatrix}
        1 & 1\\
        1 & -1
    \end{pmatrix}, \,
    Q_{Y} = \frac{1}{\sqrt{2}} \begin{pmatrix}
        1 & 1\\
        i & -i
    \end{pmatrix}
\end{align}

We can decompose the $Q_{Y}$ with next procedure.

\begin{align}
    e_{1, Y}  = \begin{pmatrix}
        1 & 0 \\
        0 & i
    \end{pmatrix} e_{1, x}, \,
    e_{-1, Y}  = \begin{pmatrix}
        1 & 0 \\
        0 & i
    \end{pmatrix} e_{-1, x} \\
    Q_Y = \begin{pmatrix}
        e_{1, Y} & e_{-1, Y}
    \end{pmatrix} = 
    \begin{pmatrix}
        \begin{pmatrix}
            1 & 0 \\
            0 & i
        \end{pmatrix} e_{1, x} &
        \begin{pmatrix}
            1 & 0 \\
            0 & i
        \end{pmatrix} e_{-1, x}
    \end{pmatrix} = 
    \begin{pmatrix}
        1 & 0 \\
        0 & i
    \end{pmatrix} 
    \begin{pmatrix}
        e_{1, X} & e_{-1, X}
    \end{pmatrix} \\
    \therefore Q_Y = \begin{pmatrix}
        1 & 0 \\
        0 & i
    \end{pmatrix}  Q_X
\end{align}
The each $Q$ matrix components are standard quantum gates, Hadamard and S gates.
\begin{itemize}
    \item Hadamard: $H = \frac{1}{\sqrt{2}}\begin{pmatrix}
        1 & 1 \\
        1 & -1
    \end{pmatrix}$
    \item S: $S = \frac{1}{\sqrt{2}}\begin{pmatrix}
        1 & 0 \\
        0 & i
    \end{pmatrix}$
\end{itemize}
Thus, we have next
\begin{align}
    X = H Z H \\
    Y = S H Z H S^\dagger
\end{align}.

From the above result we could represent $X, Y$ rotation with 
$Z$, Hadamard and S gates.

\begin{equation}
    RX(2\theta) = \exp(-i \theta X) = \exp(-i \theta ( H Z H)) = H\exp(- i\theta Z) H
\end{equation}

\begin{figure}
    \centering
    \begin{quantikz}
        &\gate{RX(\theta)}&\\
        &\gate{RY(\theta)}&\\
        &\gate{RY(\theta)}&
    \end{quantikz}
    =
    \begin{quantikz}
        &\gate{H} & \gate{RZ(\theta)} & \gate{H}& \\
        \gate{S^\dagger} &\gate{H} & \gate{RZ(\theta)} & \gate{H} & \gate{S}\\
        &\gate{S^\dagger} & \gate{RX(\theta)} & \gate{S}&
    \end{quantikz}
    \caption{Circuit representation of each rotation gates.}
    \label{fig:circuit:rotation_basic_represenation}
\end{figure}


\begin{example}[Bloch representation]
    The \begin{quantikz}
        \gate{H}\end{quantikz} and \begin{quantikz}\gate{S}
    \end{quantikz} are basis transformation matrices and 
    in Bloch sphere representation, they are the rotation 
    transformations of qubit state vector.

    \begin{center}
        \begin{tikzpicture}
% Define radius
\def\r{2}

\draw (0,0) node[circle, fill, inner sep=1] (orig) {};
%% Bloch vector
%\draw (0, 0) node[circle, fill, inner sep=1] (orig) {} -- (\r/3, \r/2) node[circle, fill, inner sep=0.7, label=above:$\vec{a}$] (a) {};
%\draw[dashed] (orig) -- (\r/3, -\r/5) node (phi) {} -- (a);

% Sphere
\draw (orig) circle (0.7*\r);

% rotation
\draw (0, 0.85*\r) node[] (z-rot-node) {};
\draw (0.85*\r, 0) node[] (y-rot-node) {};
\draw (-0.85*\r/5, -0.85*\r/3) node[] (x-rot-node) {};

\draw[-{Latex[length=1mm]}, semithick] 
    ($(z-rot-node.center) + ({-\r*cos(70)/8}, {\r*sin(70)/24})$) 
    arc
[
    start angle=110,
    end angle=430,
    x radius=\r/8,
    y radius =\r/24
];
\draw[-{Latex[length=1mm]}, semithick] 
    ($(y-rot-node.center) + ({\r*sin(70)/24}, {\r*cos(70)/8})$) 
    arc
[
    start angle=20,
    end angle=340,
    x radius=\r/24,
    y radius =\r/8
];
\draw[-{Latex[length=1mm]}, semithick] 
    ($(x-rot-node.center) + ({-\r/8}, {0})$) 
    arc
[
    start angle=200,
    end angle=480,
    x radius=\r/8,
    y radius =\r/8
];
% Axes
\draw[dashed] (orig) -- ++(\r/5, \r/3);
\draw[dashed] (orig) -- ++(-\r, 0);
\draw[dashed] (orig) -- ++(0, -\r);
    
\draw[->, thick] (orig) -- ++(-\r/5, -\r/3) node[below] (x1) {$X$};
\draw[->, thick] (orig) -- ++(\r, 0) node[right] (x2) {$Y$};
\draw[->, thick] (orig) -- ++(0, \r) node[above] (x3) {$Z$};
    
\draw[->, semithick, red] (orig) -- ++(-0.75*\r/5, -0.75*\r/3) node[left] (x0) {\tiny \color{black}$X_0$};
\draw[->, semithick, green] (orig) -- ++(0.75*\r, 0) node[above] (y0)           {\tiny \color{black}$Y_0$};
\draw[->, semithick, blue] (orig) -- ++(0, 0.75*\r) node[right] (z0)            {\tiny \color{black}$Z_0$};
\end{tikzpicture}
        \captionof{figure}{
            Bloch sphere of qubit. 
            Each rotation arrow at the end of the axis indicates 
            RX, RY, RZ rotation.}
        \label{fig:bloch-sphere-example}
    \end{center}

    In Bloch sphere representation, Hadamard gate is a $\pi$ rotation along, $\frac{1}{\sqrt{2}}(1, 0, 1)$,
    and S gate is a $\pi/2$ rotation along, $Z$ axis.
    \begin{center}
        \begin{tikzpicture}
% Define radius
\def\r{2}

\draw (0,0) node[circle, fill, inner sep=1] (orig) {};
%% Bloch vector
%\draw (0, 0) node[circle, fill, inner sep=1] (orig) {} -- (\r/3, \r/2) node[circle, fill, inner sep=0.7, label=above:$\vec{a}$] (a) {};
%\draw[dashed] (orig) -- (\r/3, -\r/5) node (phi) {} -- (a);

% Sphere
\draw (orig) circle (0.7*\r);

% rotation
\draw (0, 0.85*\r) node[] (z-rot-node) {};
\draw (0.85*\r, 0) node[] (y-rot-node) {};
\draw (-0.85*\r/5, -0.85*\r/3) node[] (x-rot-node) {};

\draw[-{Latex[length=1mm]}, semithick] 
    ($(z-rot-node.center) + ({-\r*cos(70)/8}, {\r*sin(70)/24})$) 
    arc
[
    start angle=110,
    end angle=430,
    x radius=\r/8,
    y radius =\r/24
];
\draw[-{Latex[length=1mm]}, semithick] 
    ($(y-rot-node.center) + ({\r*sin(70)/24}, {\r*cos(70)/8})$) 
    arc
[
    start angle=20,
    end angle=340,
    x radius=\r/24,
    y radius =\r/8
];
\draw[-{Latex[length=1mm]}, semithick] 
    ($(x-rot-node.center) + ({-\r/8}, {0})$) 
    arc
[
    start angle=200,
    end angle=480,
    x radius=\r/8,
    y radius =\r/8
];
% Axes
\draw[dashed] (orig) -- ++(\r/5, \r/3);
\draw[dashed] (orig) -- ++(-\r, 0);
\draw[dashed] (orig) -- ++(0, -\r);
    
\draw[->, thick] (orig) -- ++(-\r/5, -\r/3) node[below] (x1) {$X$};
\draw[->, thick] (orig) -- ++(\r, 0) node[right] (x2) {$Y$};
\draw[->, thick] (orig) -- ++(0, \r) node[above] (x3) {$Z$};
    
\draw[->, semithick, red]   (orig) -- ++(0, 0.75*\r)    node[right] (x0) {\tiny \color{black}$X_0$};
\draw[->, semithick, green] (orig) -- ++(-0.75*\r, 0)                node[above] (y0)           {\tiny \color{black}$Y_0$};
\draw[->, semithick, blue]  (orig) -- ++(-0.75*\r/5, -0.75*\r/3)               node[left] (z0)            {\tiny \color{black}$Z_0$};

\draw (-\r, \r) node[right] {$H$};
\end{tikzpicture}
        \begin{tikzpicture}
% Define radius
\def\r{2}

\draw (0,0) node[circle, fill, inner sep=1] (orig) {};
%% Bloch vector
%\draw (0, 0) node[circle, fill, inner sep=1] (orig) {} -- (\r/3, \r/2) node[circle, fill, inner sep=0.7, label=above:$\vec{a}$] (a) {};
%\draw[dashed] (orig) -- (\r/3, -\r/5) node (phi) {} -- (a);

% Sphere
\draw (orig) circle (0.7*\r);

% rotation
\draw (0, 0.85*\r) node[] (z-rot-node) {};
\draw (0.85*\r, 0) node[] (y-rot-node) {};
\draw (-0.85*\r/5, -0.85*\r/3) node[] (x-rot-node) {};

\draw[-{Latex[length=1mm]}, semithick] 
    ($(z-rot-node.center) + ({-\r*cos(70)/8}, {\r*sin(70)/24})$) 
    arc
[
    start angle=110,
    end angle=430,
    x radius=\r/8,
    y radius =\r/24
];
\draw[-{Latex[length=1mm]}, semithick] 
    ($(y-rot-node.center) + ({\r*sin(70)/24}, {\r*cos(70)/8})$) 
    arc
[
    start angle=20,
    end angle=340,
    x radius=\r/24,
    y radius =\r/8
];
\draw[-{Latex[length=1mm]}, semithick] 
    ($(x-rot-node.center) + ({-\r/8}, {0})$) 
    arc
[
    start angle=200,
    end angle=480,
    x radius=\r/8,
    y radius =\r/8
];
% Axes
\draw[dashed] (orig) -- ++(\r/5, \r/3);
\draw[dashed] (orig) -- ++(-\r, 0);
\draw[dashed] (orig) -- ++(0, -\r);
    
\draw[->, thick] (orig) -- ++(-\r/5, -\r/3) node[below] (x1) {$X$};
\draw[->, thick] (orig) -- ++(\r, 0) node[right] (x2) {$Y$};
\draw[->, thick] (orig) -- ++(0, \r) node[above] (x3) {$Z$};
    
\draw[->, semithick, red] (orig) -- ++(0.75*\r, 0) node[above] (x0) {\tiny \color{black}$X_0$};
\draw[->, semithick, green] (orig) -- ++(0.75*\r/5, 0.75*\r/3) node[above] (y0)           {\tiny \color{black}$Y_0$};
\draw[->, semithick, blue] (orig) -- ++(0, 0.75*\r) node[right] (z0)            {\tiny \color{black}$Z_0$};

\draw (-\r, \r) node[right] {$S$};
\end{tikzpicture}
        \captionof{figure}{
            Applied axis transformation by Hadamard and S gates.
            The left is a Hadamard transformation and the right is a S gate transformation.}
        \label{fig:h-s-rotation}
    \end{center}
\end{example}


\subsection{N-qubit system}

%The $\Delta t$ and norm of the Hamiltonian, $|\mathcal{H}|$ act as a rotation angle, 
%and the direction of the $\mathcal{H} \cdot \mathbf{\hat{\sigma}}$ vector acts as a rotation axis 
%on phase space of the qubit system. We will see details in the next section.
%
%A Pauli-basis of 
%$n$-qubit system 
%
%Most of the properties of Pauli-matrices are preserved in 
%$n$-fold Pauli-basis.

In the previous section, we briefly discussed basic theorems for approximating the evolution 
process with local terms, and provided a practical example with a single qubit. 
Now, we extend the construction to N-qubit systems. 
Achieving a higher-dimensional version of the operator is not possible 
by simply repeating the single-qubit operator across the entire system\footnote{Hadamard is possible, but very special case.}. 
Sometimes we only have approximation representation in higher case.
Luckily, evolution circuit is not. 
We can construct a solid structure of the circuit for multi-qubit systems,
so that we can find an exact $N$ qubit circuit 
for multi-qubit Pauli operators\footnote{We don't have to consider Solovay-Kitaev process.}.

\subsubsection{Multi qubit Pauli matrices}

\textit{Pauli strings}: Pauli string is a representation of multi-qubit Pauli operator with combination of single Pauli operators, $I, X, Y, Z$.
For example, Pauli string of $P = I \otimes X \otimes Y \otimes X$ is $IXYX$.
\index{Pauli string}
We usually called $n$-fold Pauli string when we want to emphasize the dimension.

In the previous exercise, we stated that 
the evolution on single Pauli term is same with the rotation 
along a specific axis in Hilbert-space, corresponding to the Pauli terms, $I, X, Y, Z$.
Same statement hold for $n$ qubit wires, since tensor product of Pauli terms also hold
same properties of single terms\footnote{That is why they are called by Pauli string.}.
Therefore, we can use same formula here, $R(\theta; P_i) = \cos(\theta) I+ i \sin(\theta)P_i$.


\subsubsection{Evolution as conditional rotation}

Recall the $RZ$ gate as matrix form, 

\begin{equation}
    Z = \begin{bmatrix}1 & 0 \\ 0 & -1 \end{bmatrix}, \, RZ(\theta) = \begin{bmatrix} e^{-i \theta} & 0 \\ 0 & e^{i \theta} \end{bmatrix}
\end{equation}

We can interpret the gate as conditional phase rotation by the states.

\begin{equation}
    RZ(\theta) = \begin{cases}
        |0 \rangle \rightarrow e^{- i\theta} |0\rangle \\
        |1 \rangle \rightarrow e^{  i\theta} |1\rangle \\
    \end{cases}
\end{equation}

Now observe the next Pauli term. 
We can use same convention what we did in 1 qubit Pauli Hamiltonian.

\begin{equation*}
    Z_1 Z_2 = \begin{bmatrix}
        1 & 0 & 0 & 0 \\
        0 & -1 & 0 & 0\\
        0 & 0  & -1 & 0\\
        0 & 0 & 0 & 1 
    \end{bmatrix},
    %I_1 Z_2  = \begin{bmatrix}
    %    1 & 0 & 0 & 0 \\
    %    0 & 1 & 0 & 0\\
    %    0 & 0  & -1 & 0\\
    %    0 & 0 & 0 & -1 
    %\end{bmatrix}
\end{equation*}

\begin{equation}
    \label{eq:Z1Z2_rot}
    \exp(- i \theta Z_1 Z_2) = \begin{cases}
        |\psi \rangle \rightarrow e^{- i\theta} |\psi\rangle & \mbox{if } \psi = 00 \mbox{ or } 11\\
        |\psi \rangle \rightarrow e^{  i\theta} |\psi\rangle & \mbox{if } \psi = 01 \mbox{ or } 10\\
    \end{cases}
\end{equation}

See Fig \ref{fig:evol_cir_z1z2}. It is a corresponding evolution circuit of $Z_1Z_2$ Hamiltonian. 
\begin{marginfigure}
    \centering
    \begin{quantikz}
        &\ctrl{1}& & \ctrl{1}&\\
        & \targ{}& \gate{RZ(2 \Delta t)}& \targ{} &\\
    \end{quantikz}
    \caption{Evolution circuit example of $Z_1Z_2$.}
    \label{fig:evol_cir_z1z2}
\end{marginfigure}

\subsubsection{CNOT gate}

How the CNOT gate and Rotation Z gate could implement $\exp(-i t Z_1 Z_2)$
operation? First, look at definition of CNOT gate, it flips the target 
qubit along X axis, by the control qubit state. 

\begin{center}
    \begin{quantikz}
        &\ctrl{1}&\\
        & \targ{}  &\\
    \end{quantikz} = 
    $\begin{bmatrix}
        1 & 0 & 0 & 0 \\
        0 & 1 & 0 & 0 \\
        0 & 0 & 0 & 1 \\
        0 & 0 & 1 & 0 \\
    \end{bmatrix} = (|0 \rangle \langle 0| \otimes I + |1\rangle \langle 1| \otimes X)$
    \captionof{figure}{CNOT gate}
\end{center}

Meanwhile, in basis view point, we can treat them as 
a mapping operator as 
$\{|00\rangle, |11\rangle\} \rightarrow |0\rangle$, and 
$\{|01\rangle, |10\rangle\} \rightarrow |1\rangle$.
See Fig \ref{fig:CNOT_basis}.

\begin{equation*}
    | x_1 \rangle \otimes |x_2 \rangle \rightarrow | x_1 \rangle \otimes | x_1 \oplus x_2 \rangle 
\end{equation*}

\begin{marginfigure}
    \centering
    \begin{quantikz}
        \lstick{$|{x_1}\rangle$}&\ctrl{1}&\rstick{$|x_1 \rangle$}\\
        \lstick{$|{x_2}\rangle$}&\targ{ }&\rstick{$|x_1 \oplus x_2 \rangle$}
    \end{quantikz} 
    \caption{CNOT as basis change operator.}
    \label{fig:CNOT_basis}
\end{marginfigure}

Therefore, the $I \otimes RZ(2 \Delta t)$ works of $\exp(-i \Delta t Z_1 Z_2)$ in Eq (\ref{eq:Z1Z2_rot}).

\begin{exercise}
    Is 
    \begin{quantikz}
        &\targ{}&  \gate{RZ(2 \Delta t)} & \targ{}&\\
        & \ctrl{-1}& & \ctrl{-1} &\\
    \end{quantikz} circuit identical to Fig \ref{fig:evol_cir_z1z2}? Why?
\end{exercise}

%Here we introduce the good framework to analysis the axis of the wires in each 
% Bloch sphere representation.

%\begin{example}
%    Lamor-precession of single particle in uniform magnetic field, $B$ can be formulated as next,
%
%    \begin{equation}
%        H = \mathbf{\mu} \mathbf{B}
%    \end{equation}
%\end{example}

\subsubsection{CX structure}%--------------------

\begin{figure}[!ht]
    \centering
    \begin{quantikz}
        &\ctrl{3}&           &          &                       &         &        & \ctrl{3}&\\
        &        &\ctrl{2}   &          &                       &         &\ctrl{2}&         &\\
        &        &           & \ctrl{1} &                       &\ctrl{1} &        &         &\\
        &\targ{} &\targ{}    & \targ{}  &\gate{RZ(2 \Delta t)}&\targ{}  &\targ{} &\targ{}  & 
    \end{quantikz}
    \caption{$\Delta t$ evolution circuit of $H=ZZZZ$ Hamiltonian.}
    \label{fig:ZZZ_evolve_1}
\end{figure}


In other manuscripts, next type circuit is also common.

\begin{figure}[!ht]
    \centering
    \begin{quantikz}
        &\ctrl{1}&           &          &                     &         &        &\ctrl{1}&\\
        &\targ{} &\ctrl{1}   &          &                     &         &\ctrl{1}&\targ{} &\\
        &        &\targ{}    & \ctrl{1} &                     &\ctrl{1} &\targ{} &        &\\
        &        &           & \targ{} &\gate{RZ(2 \Delta t)} &\targ{}  &        &        &
    \end{quantikz}
    \caption{$\Delta t$ evolution circuit of $H=ZZZZ$ Hamiltonian 2nd type.}
    \label{fig:ZZZ_evolve_2}
\end{figure}

The Fig \ref{fig:ZZZ_evolve_1} is called by \textit{fountain} type, 
and Fig \ref{fig:ZZZ_evolve_2} is called by \textit{chain} type.

\begin{exercise}
    Show that the above two circuits are identical. 
    Do not use a matrix representation. \textit{Hint: See a property of direct sum or bitwise XOR, both are identical in binary vector}.
\end{exercise}

If you construct the circuit in matrix form,
you can verify that those two circuits are identical.
In the end, the mixture of two type also identical
and the position of the $RZ$ gate is also freely determined
by the situation. 
The only thing we need is a CNOT gate path
visit each i-th qubits in the Hamiltonian at once\footnote{Considering qubit rearangement.}.

\begin{center}
\begin{quantikz}
    &\ctrl{3}&           &          &                      &             &           &\ctrl{3}&\\
    &        &\targ{}    & \targ{}  & \gate{RZ(2 \Delta t)}& \targ{}     & \targ{}   &       &\\
    &        &           & \ctrl{-1}&                      & \ctrl{-1}   &           &       &\\
    &\targ{} &\ctrl{-2}  &          &                      &             & \ctrl{-2} &\targ{}&
\end{quantikz}
\end{center}

\subsubsection{Basis transformation}

We only analyzed $I, Z$ strings. 
General strings consist of $I, Z, X, Y$ yields next circuit for evolution operator.


\begin{equation}
    \exp(-i \Delta t XYZZ)
\end{equation}

\begin{figure}[!ht]
    \centering
    \begin{quantikz}
       &\gate{H}         &          &\ctrl{1}&           &          &                     &         &        &\ctrl{1}& \gate{H}&         &\\
       &\gate{S^\dagger} & \gate{H} &\targ{} &\ctrl{1}   &          &                     &         &\ctrl{1}&\targ{} & \gate{H}& \gate{S}&\\
       &                 &          &        &\targ{}    & \ctrl{1} &                     &\ctrl{1} &\targ{} &        &         &         & \\
       &                 &          &        &           & \targ{} &\gate{RZ(2 \Delta t)} &\targ{}  &        &        &         &         & 
    \end{quantikz}
\end{figure}

\begin{example}
    Understanding the circuit with graphic notation of basis.

\end{example}


\subsection{Implementation with Phase operator}

The evolution circuit was a conditional phase rotation by 
the system state. There is a Phase operator in quantum 
computer.

\begin{equation*}
    \mbox{P}(\theta) = \begin{bmatrix}
        1 & 0 \\
        0 & e^{i \theta}
    \end{bmatrix}, 
    \mbox{RZ}(\theta) = \begin{bmatrix}
        e^{-i \theta/2} & 0 \\
        0 & e^{i \theta/2}
    \end{bmatrix}
\end{equation*}

\begin{marginfigure}
    \begin{quantikz}
        &\ctrl{1}&\\
        &\gate{P(\theta)}&
    \end{quantikz}
\end{marginfigure}

They are same in single qubit system, however, in multi-qubit system a rotation z gate and 
a phase gate work differently because of the local phases.
Even though, we can implement the same operation
with phase gates without rotation z gates.

\begin{center}
    \begin{quantikz}
        & \ctrl{1} &                  & \ctrl{1} & \\
        & \targ{}  & \gate{RZ(\theta)} & \targ{} &
    \end{quantikz}
    = $e^{-i \theta/2} \begin{bmatrix}
    1 & 0 & 0 & 0 \\
    0 & e^{i \theta} & 0 &0\\
    0 & 0 & e^{i \theta}& 0\\
    0 & 0 & 0   & 1
    \end{bmatrix}$
    \\
    \begin{quantikz}
        & \gate{P(\alpha)} & \ctrl{1} & \\
        & \gate{P(\beta)}  & \gate{P(\gamma)} &
    \end{quantikz}
    = $\begin{bmatrix}
    1 & 0 & 0 & 0 \\
    0 & e^{i \beta} & 0 &0\\
    0 & 0 & e^{i \alpha}& 0\\
    0 & 0 & 0   & e^{i (\alpha + \beta) \gamma} 
    \end{bmatrix}$
\end{center}

If $\alpha = \beta = \theta$ and $\gamma = -2 \theta$, the two circuit are identical
without considering global phase.
Similarly, we can arange RZ gates to construct the evolution circuit 
without CX gate, but using controlled-RZ gate.

\begin{exercise}
    Show that the RZ and controlled-RZ could generate $ZZ, IZ, ZI$ evolution gate 
    on 2 qubit circuit. 
\end{exercise}


Note that, when you work with gate model,
it is very common case that using an ancilla registers
to reduce the circuit depth, or to implement some algorithms.
During the process, you must take care of the relative 
phase between the original register and the ancilla register.
Sometimes it does not affect but before the calculation 
you must ensure the consistency.
Think about Ahrno-Bohm effect, it is not a strange 
effect that the local phase difference affect the overall
result.


\section{Additional Note}


\subsection{Clifford Group}

The evolution circuit comprise large 3 gate sets.

\begin{itemize}
    \item Rotation gates: $RZ$
    \item Entaglement gates: $CNOT$
    \item Basis Transformation gates: $H, S$
\end{itemize}


For example, using two rotation gates, $RZ, RX$
we can eliminate $H, S$ gates in the evolution circuit.
On the other hand, we can use 3 rotation gates
without those basis change gates. 


% 2 qubit gates

% 3 -qubit gates



\subsection{Pauli-Frame}

As a rotation, geographical representation only work for 
1-qubit case.
In $N$-qubit case, there is no representation method 
including all quantum information and state.
Same hold for each rotation gate in Trotterization circuit.
We accept the circuit in Fig () is a rotation around axis \textbf{nemo}.
What about the next figure?


Back to the conditional rotation interpretation of 
the evolution matrix.



There is an additional physical meaning in Pauli Frame.
It represents a mutually commuting axes set that we 
can manipulate on the circuit.
