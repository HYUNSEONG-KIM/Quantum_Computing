\chapter{Tunneling Phenomenon}

Quantum tunneling is one of non-classical phenomenon in real-world with entanglement and wave-particle duality.
Suppose that the free particle of $m, p$ mass and momentum pass through a potential varrier, $V$ in 1-dimension.

\section{Introduction to tunneling}

\begin{figure}
    \caption{}
    \label{fig:tunneling}
\end{figure}

The solution of the Schr$\ddot{\mbox{o}}$dinger  equation yields,

\begin{eqnarray}
    \psi(x) = A \exp(\pm ikx)
\end{eqnarray}


The tunneling effect arises in the situation when $E <V$. 
In classically, at the $b<x$, the existence possibility of the particle is zero.
However, it is not in quantum mechanics


\subsection{alpha-decay}
The tunneling effect successfully explained the \textit{alpha}-decay phenonmenon by George Gamow, and Condon and Gurney, indepedently.
In the Columb potential of the nuclear, 
the alpha particle cannot obtain the enough energy 
to escape the nuclear binding force.
However, in many experiment and observation alpha particle emission was occured

\section{Tunneling time problem}

\begin{definition}{\textbf{Tunneling time problem}}

    How many time does it take to pass through the potential region, when the tunneling effect is occurred?
\end{definition}

This simple question has not been solved yet.

\section{Criticisims on tunneling time}

The major problem is that there does not exist about some common definition of \textit{time} in quantum mechanics.
Time is just a parameter in quantum mechanics, meanwhile, in general relativity, spacetime dynamically interacts with matters.  

Imaginary time evolution problem: Imaginary number has no proper order by the definition. 
How can we define proper flow direction of time in imaginary numbers?

There are some definitions of tunneling time.
The concepts have their own criticisims about how they well explain the tunneling time problem, 
however, they are practically used in required field, such electrodynamics.

\begin{enumerate}
    \item Dwell-time
    \item Wignet-time
    \item Larmor-time
\end{enumerate}

A proper time observable, $T_c$, should satisfy the following properteis:

\begin{enumerate}
    \item On average $T_c$ should estimate te elaspsed time $\tau$.
    \item The variance in a measurement of $T_c$ should be independent of the elapsed time $\tau$.
\end{enumerate}

\subsection{Classical approach}

In the classsic scale, the potential is simply a hill. 

\subsection{Measurement- Experimental}

\subsubsection{Larmor Precession}

The spin is rotated in the uniform magentic field, $B_0 \hat{z}$ with constant frequency $\omega$. 
Such phenomenon is called \textit{Larmor precession} and the frequency is called \textit{Larmor frequency}.
If the particle is 

\subsection{Hartman effect and Superluminal velocity}

Hartman effect


\section{Simulation and Measurement on Quantum circuit}

Consider Larmor precession implementation on quantum circuit. 
The spatial propagation of 1 dimension is well established at section \ref{sec:1dim_particle}.
If we want to implement Larmor precession, 
we can add an additional qubit to act as a spin of the free particle,
and modifying the Hamiltonian adding spiner term. 
Does it really a good simulation? Why we do a calculation and simulate the phenomenon on circuit?
We want to obtain a complex system behavior through the simulation, 
however, designing an affecting circuit to such qubit by time-evolution effect 
is itself a problem we want to solve. 
In experiment, it could be a good clock system for the tunneling phenomenon.
Meanwhile, in simulation, we cannot get a meaningful information from the model.

In addition, it is just a extracting time\footnote[4]{Maybe events} information from the system register.
Why it must be Larmor time depending on only a specific potential; magenetic potential.
Tunneling is an universal phenonmenon that does not depends on the types of potential.
A good clock simulator must act on the general situation whatever potnetial type it is.

\subsection{Page Wootter Mechanics}

The previous definitions of clock and time of tunnleing effect were based on the wave function, 
probability current and their change by the time.

We are focusing on the situation of \textbf{the event occurred} stuation.
The particle occurrence proability of the given space is closely related with the event occurrence.
However, there is a gap between the stuations. 
Unfortunately, in the common Schr$\ddot{\mbox{o}}$dinger  equation cannot handle the case of our attention.
It is because the Schr$\ddot{\mbox{o}}$dinger  equation didn't consider the relativity, so that the time is \textit{not an observable quantity}.
The dynamical interaction of space time with matter is described by general relativity.
Related with gravity and quantum system. 

However, Dirac/canonical quatization of gravity yields \textit{Frozen Formalism}\footnote{It was impeorted from , Smith, Aleander, "The Page-Wootters formalism: Where are we now?", 10.48660/22010079}. 
General realtivity can be represented with Hamiltonian form as Eq (\ref{eq:ge_hamiltonian}).

\begin{align}
    H_{GR}[\gamma , P] = 16 \pi G  G_{abcd}[\gamma] P^{ab} P^{cd} + V{\gamma} + \sqrt{\gamma} \rho =0 \label{eq:ge_hamiltonian}\\
    H_{GR} | \Psi \rangle = 0\\
    i \frac{d}{d t} | \Psi \rangle = H_{GR} | \Psi \rangle = 0
\end{align}

What does the above equation means? The state vector $| \Psi \rangle$ is frozen which means there is no evolution.
This is gravitation quantization problem related to the problem of time.

However, recently, there is a re-splotlighted perspective of quantum time in the field which has been known as 
\textit{Page-Wootter formalizm}. In 1983, Page and Wootters presented extended version of the equation\cite{PhysRevD.27.2885}. 
This is called PaW mechanics.
PaW mechanics treat the time as a quantum degree of freedom in specific Hilbert space, and 
in the viewpoint, time flow is occurred from the entanglement between such space and the remained physical system.
There were some critisim about the formalization of PaW mechanics;proper propagator, ... .
Giovannetti technically reformalize the PaW structure and showed PaW mechanics can be extended 
to give the time-independent Schr$\ddot{\mbox{o}}$dinger  equation\cite{PhysRevD.92.045033}.


\subsubsection{Evolution as a basis change}

The evolution is an unitary transformation, and 
every unitary transformation is a kind of basis transformation.
Even the overall state is frozen by the time. 
The basis change does not affect the static state,
however, locally there could be a dynamics.

\subsubsection{Structure of PaW mechancis}

The system, $\mathfrak{H}$ is consist of Hilbert space of the ordinary system, $\mathcal{H}_s$ and ancillary Hilbert space, 
$\mathcal{H}_T$, of clock or time.

\begin{eqnarray}
    \mathfrak{H} = \mathcal{H}_T \otimes \mathcal{H}_S
\end{eqnarray}

\begin{eqnarray}
    \hat{\mathbb{J}}  = \hbar \hat{\Omega} \otimes \mathbf{1}_S + \mathbf{1}_T \otimes \hat{H}_S \\
    |\Psi \rangle \rangle = \int dt | t \rangle_T \otimes | \psi(t) \rangle_S \\
    \hat{\mathbb{J}} | \Psi \rangle\rangle = 0 
\end{eqnarray}

From unitary time evolution operator on the system, $\hat{U}_S(t)$ we can construct a proper evolution of the system.

\begin{eqnarray}
    \mathbb{U} &=& \int dt | t \rangle_T \langle| \otimes \hat{U}_S(t) \\
    &=& \hat{U}(\hat{T}) = \exp(- i \hat{T} \otimes \hat{H}_s /\hbar) 
\end{eqnarray}


The main benefit of the PaW mechanics is that the time is an observable quantity in the mechanics and 
we can obtain a basyian probability on time-dimension about the occurred event following Boron's rule.

\begin{equation}
    P(A |B) = \frac{\mbox{tr}(P_A P_B \rho P_B) }{\mbox{tr}(P_B \rho P_B)}
\end{equation}

In the approach, the author treating the quantum tunneling time on the analysis of \textit{arrival time}.

The goodness of the PaW mechanism is that it can be directly implemented on quantum circuit by ancilla qubit registers.
The pure state is well-suited with qubit network and 
