\chapter{Variational method}

It is considered practically useful method in
NISQ era\footnote{Noisy intermediate scale quantum era}.


Review \cite{cerezo_variational_2021}.
\section{Introduction}

\subsection{Variatiaonal Principle}

\subsection{Circuit as ansatz}

The term \textit{ansatz} means an assumption 
in mathematics or physics to solve the given problem.
In the physics, it is commonly refer a model representing 
or containing the solution of the problem.
The statement of \textit{Is the ansatz proper to solve the problem?}
refers the whether is it possible that "The solution could be represented with the model"
or, in optimization problem, "The ansatz achive the desire solution".

Using a paramerized circuit, we can treat the quantum circuit 
as a neural network in machine leanring. 
However, if the result is not bounded, 
the optimization is meaningless for any kinds purpose.
That is why we need a \textit{Variational principle}.
Since, quantum circuit is a kind of quantum system and their
ground state and energy is limited in finite value.
Same statement holds for parameterized circuit, there always 
exists a ground state energy\footnote{Not a state. There could be degenerated states.}.

Whatever energy we measure from quantum circuit, it is always higher or equal with the ground state.

\begin{equation*}
    E_{ground} \leq \langle \psi | H | \psi \rangle
\end{equation*}


We can optimize the circuit without any worry about their convergence.


\begin{principle}
    \textbf{Variational Principle}

    \begin{equation}
        E_{g} \leq \langle \psi | H | \psi \rangle \leq E_{\mbox{max}}
    \end{equation}
\end{principle}

\subsection{Universality of Circuit}

One of the strong background of the neural network model is 
\textit{Universal Approximation Theorem}. 

How about quantum circuit? Do we have any those type of theorem?
The answer is yes!. Quantum circuit is also a universal approximator.
In addition, it is more powerful. Even with a single qubit we can achieve the property\cite{PhysRevA.104.012405-Universal-approxi}.

\section{}

\section{Ansatzs}

\subsection{Hardware Efficient}

\subsection{Evolution Based}


\section{Application}

\subsection{QUBO}

\subsection{Evolution}

\subsubsection{Real time}

\subsubsection{Imaginary time}

\section{Implementation}

\subsection{Pennylane}

\subsection{Qiskit}

\bibliographystyle{unsrt}
\bibliography{vqe}

