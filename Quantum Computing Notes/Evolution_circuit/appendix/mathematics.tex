\chapter{Miscellaneous Mathematics}

\section{Tensor Product}

\begin{definition}{\textbf{Tensor product}}
    The given two vector space, $V$ and $W$ over the field $\mathbb{F}$,
    a tensor product is a bi-linear mapping with notation, $\otimes$, such that

    \begin{equation}
        \otimes: V \times W \rightarrow \mathcal{U} = V \otimes W
    \end{equation}
\end{definition}

The generated space $\mathcal{U} = V \otimes W$ also be a 
vector space over field $\mathbb{F}$. The tensor product generates a larger vector space 
with two-given vector spaces. 

\begin{theorem}{\textbf{Properties of Tensor producted space}}

    \begin{itemize}
        \item Tensor product of the spanning sets of the each VS is a spanning set of the producted space.
        \item For finite VS $\dim(V) = n, \dim(W) = m$, $\dim(V \otimes W) = n \cdot m$.
        \item Dual space of the tensor producted space is a tensor product of dual spaces of each VS.
    \end{itemize}
\end{theorem}

In matrix space, $\mathbf{M}_{}(\mathbb{C})$ is form a Hilbert-space with Hilbert-Schmidt inner product.

\begin{definition}{\textbf{Hilbert-Schmidt Inner Product}}
    For the given two matrices, $A, B$, their inner product is defined as 

    \begin{equation}
        \langle A | B \rangle = \mbox{Tr}(A^\dagger B)
    \end{equation}
    
\end{definition}

A typical tensor product of matrix space is a \textit{Kronecker product}.

\begin{definition}{\textbf{Kronecker Product}}
    \begin{equation}
        A \otimes B = \begin{pmatrix}
            a_{11} B & \cdots & a_{1m} B\\
            \vdots & \ddots & \vdots \\
            a_{n1} B & \cdots & a_{nm} B
        \end{pmatrix}
    \end{equation}
\end{definition}

\section{Tensor product representation of quantum circuit}




\section{Lie-algebra}
Lie-group and Lie-algebra is a mathematical formulation to express the group sturcture in VS for a convience.

